\section[Понятие и классификация СППР]
{ПОНЯТИЕ И КЛАССИФИКАЦИЯ СППР}

С течением времени понятие СППР несколько раз изменялось и разделялось на
несколько равнозначимых определений. На сегодняшний день СППР можно дать
следующее определение.

\textit{Система поддержки принятия решений (СППР, англ. Decision Support System, DSS)} ---
автоматизированная система (в основном компьютерная программа), целью которой
является помощь людям, принимающим какие-либо решения.

Система поддержки принятия решений предназначена для поддержки многокритериальных
решений в сложной информационной среде. При этом под многокритериальностью понимается
тот факт, что результаты принимаемых решений оцениваются не по одному, а по совокупности
многих критериев, рассматриваемых одновременно. Информационная сложность определяется
необходимостью учета большого объема данных, обработка которых без помощи
современной вычислительной техники практически невыполнима. В этих условиях число
возможных решений, как правило, весьма велико, и выбор наилучшего из них,
без всестороннего анализа может приводить к ошибкам.

Система поддержки решений СППР решает две основные задачи:
\begin{itemize}
  \item выбор наилучшего решения из множества альтернатив (оптимизация);
  \item упорядочение альтернатив по предпочтительности (ранжирование).
\end{itemize}

В обеих задачах первым и наиболее принципиальным моментом является выбор
совокупности критериев, на основе которых в дальнейшем будут оцениваться
и сопоставляться альтернативы.


\subsection{Классификация СППР}

Системы поддержки принятия решений можно классифицировать различным образом.
Рассмотрим классификацию по типу взаимодействия с пользователем, по способу
поддержки и по сфере использования.

По типу \textbf{взаимодействия с пользователем} выделяют три вида СППР:

\begin{itemize}
  \item \textit{пассивные} помогают в процессе принятия решений, но не могут выдвинуть конкретного предложения;
  \item \textit{активные} непосредственно участвуют в разработке правильного решения;
  \item \textit{кооперативные} предполагают взаимодействие СППР с пользователем.
    Выдвинутое системой предложение пользователь может доработать, усовершенствовать,
    а затем отправить обратно в систему для проверки. После этого
    предложение вновь представляется пользователю, и так до тех пор,
    пока он не одобрит решение.
\end{itemize}

По \textbf{способу поддержки} различают:
\begin{itemize}
  \item \textit{модельно}-ориентированные СППР, используют в работе доступ к статистическим,
    финансовым или иным моделям;
  \item СППР, основанные на \textit{коммуникациях}, поддерживают работу двух и более пользователей,
    занимающихся общей задачей;
  \item СППР, ориентированные на \textit{данные}, имеют доступ к временным рядам организации.
    Они используют в работе не только внутренние, но и внешние данные;
  \item СППР, ориентированные на \textit{документы}, манипулируют неструктурированной информацией,
    заключенной в различных электронных форматах;
  \item СППР, ориентированные на \textit{знания}, предоставляют специализированные решения проблем,
    основанные на фактах.
\end{itemize}

По \textbf{сфере использования} выделяют:
\begin{itemize}
  \item общесистемные;
  \item настольные.
\end{itemize}

Общесистемные работают с большими СХД и применяются многими пользователями.
Настольные являются небольшими системами и подходят для управления с
персонального компьютера одного пользователя.

\pagebreak