\section*{ВВЕДЕНИЕ}
\addcontentsline{toc}{section}{Введение}

\pagebreak

% С течением времени в обществе происходит переоценка ценностей,
% смена мыслительных парадигм, вследствие чего происходит
% преображение и переосмысление существующих теорий, а также философских направлений.
% Фундаментальная теория Карла Макрса не является исключением и с течением времени
% выражается в таких движениях как неомарксизм и постмарксизм.

% Предметом исследования является течения неомарксизма и постмарсизма в двадцатом
% веке. Объектом исследования является неомарксизм и постмарксизм.

% Работа представляется собой два взаимосвязанных раздела.
% В первом разделе описываются предпосылки возникновения неомарсизма и постмарсизма,
% даётся определение этим понятиям, рассматриваются их ценности.
% Во втором разделе приводятся основные представители неомарксизма и постмарксизма.
% Целью данной работы является рассмотрение терминов неомарксизм и постмарксизм.

% Изучение неомарксизма, постмарксизма, как и любого другого философского направления,
% помогает понять взгляды представителей исследуемой эпохи. В работе рассмотрены
% как изменялись взгляды людей с течением времени, начиная с периода классической
% марксисткой теории до постмарксизма.
