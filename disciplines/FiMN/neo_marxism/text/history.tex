\section[Предпосылки возникновения послемарксистских взглядов]
{ПРЕДПОСЫЛКИ ВОЗНИКНОВЕНИЯ \\ ПОСЛЕМАРКСИСТКИХ ВЗЛЯДОВ}

Марксизм, как философское, экономическое, социально-политическое учение,
формируется в XIX веке в период становления и утверждения на Западе
индустриального общества. Традиционно считается, что определяющее познавательное
значение в марксизме имеют материалистическое понимание истории, теория
прибавочной стоимости, учение о классовой борьбе и пролетарской революции.
Однако творческое наследие Карла Марса настолько разнообразно, что уже более
чем полтора столетия влияет на развитие общества. Споры о роли марксизма как
актуального направления социального познания носят неоднозначный характер.
И если на Западе, несмотря на разброс оценочных суждений, марксизм воспринимается
как неотъемлемая часть исторического <<пейзажа>>, то на постсоветском пространстве
признание архаичности марксизма привело к его избирательному изъятию из школьной
и университетской программ. Проблема имеет только идеологическое, но и
теоретико-методологическое значение. Марксизм является неотъемлемой частью
социально-политических наук, и его упущение из методологического обихода
приводит к погрешности и ущербности социального познания~\cite{article_neomarxism_today_and_tomorrow}.

Марксистский дискурс не прекращался с момента издания Манифеста коммунистической
партии (1848 г.) и был связан не только с реальной практикой рабочего движения.
В начале XX века марксизм повлиял на различные направления академических исследований.
Это влияние оказалось таким существенным, что само учение было воспринято не только
как идеология революционной борьбы, но и как метод познания социальности.
Как писал российский ученый, профессор Вячеслав Семёнович Стёпин, марксизм
<<подобен разросшемуся дереву, каждая ветвь которого выступает в качестве особого течения,
аспекта, толкования марксистских идей и принципов, попыток осмыслить и переосмыслить
их под углом зрения накапливаемого исторического опыта>>~\cite{era_of_change}.

Одной из таких <<веток>> теории марксизма стал \textbf{неомарксизм}, призванный быть
новым импульсом развития социалистических идей. Принципиальным отличием неомарксизма
от его предтечи является изменение представления о соотношении
роли международных и внутриобщественных отношений.
Если в марксовом понимании международные отношения носили «вторичный»
или даже «третичный» характер по отношению к внутриобщественным,
то последователи Маркса отходят от этих принципов~\cite{article_socialism}.

Буржуазная мысль не впервые переживает <<возрождение>> идей Маркса ---
новое <<открытие>> для себя его теоретического значения. Нечто очень близкое к тому,
что в этой связи происходило в 1980-х годах на Западе, имело место в 20 -- нач. 30-х годов.
При этом весьма существенна одна хронологическая деталь. Тогдашний <<марксистский
ренессанс>> возник под влиянием Октябрьской революции 1917 года. А кульминационного
пункта он достиг в период мирового экономического кризиса. В условиях этого
<<ренессанса>> и сложилось окончательное своеобразное течение буржуазной мысли,
получившей впоследствие название <<неомарксизма>>~\cite{neomarxism}.

По мнению кандидата политических наук В. С. Михайловского, можно выделить более 120
основных теоретиков неомарксизма. Для всех учёных-неомарксистов характерно
критическое отношение как к капиталистической системе, так и к советскому
варианту социализма с его марксистско-ленинской идеологией~\cite{stepin1999}.

Существуют несколько этапов развития неомарксизма.

\textit{Первый этап} (период между Первой и Второй мировыми войнами) характеризуется
возникновением интеллектуальными разработками неомарксистских теоретиков
в рамках коммунистических партий и Института социальных исследований
во Франкфурте-на-Майне.

\textit{Второй этап} (1941 -- 1968 гг.) отличается интеллектуальным расцветом
франкфуртской школы и характеризуется <<околопартийной>> аналитической
деятельностью теоретиков неомарксизма.

\textit{Третий этап} (после 1968 г.) связан с тем, что доминирующим становится
университетское развитие неомарксизма, к учению повышается интерес со стороны
Северной и Южной Америки, Центральной и Юго-Восточной Европы~\cite{article_neomarxism_today_and_tomorrow}.

\textit{Последней стадией эволюции} неомарксизма стал международный \textbf{постмарксизм},
получивший развитие во второй половине 1960-х годах в Европе. Для постмарксизма,
вписывающегося в контекст постмодернизма, совершенно неактуальным
стал водораздел между <<западным>> и <<восточным>> марксизмом.
Толчком к возникновению постмарксизма стало восстание студентов в 1968 году,
которые дискредитировали взгляды <<традиционных левых>>, что повлекло за собой
необходимость поиска новых путей развития.

% именно в том порядке, в котором они вызывали пытались его захватить~\cite{wiki_semaphore}.

