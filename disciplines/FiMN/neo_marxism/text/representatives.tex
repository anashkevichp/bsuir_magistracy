\section[Основные представители неомарксизма и постмарксизма]
{ОСНОВНЫЕ ПРЕДСТАВИТЕЛИ \\ НЕОМАРКСИЗМА И ПОСТМАРКСИЗМА}

Поначалу, понятие неомарксизма не имело каких-либо чётких очертаний. Это было,
скорее, специфическим настроением умов, захватившим определённые круги западной
интеллигенции --- главным образом леворадикальной ориентации. Грани между
неомарксизмом и иными течениями буржуазной мысли об обществе были настолько
неопределёнными и подвижными, что подчас очень трудно было сказать, является ли
тот или иной автор, использующий марксисткую фразеологию, неомарксистом или
не является. К тому же неомарксизм играл весьма различные роли в философии
и политическом развитии теоретиков, имеющих к нему то или иное отношение.

На сегодняшний день, основополагающими трудами неомарксисткой теории являются
книга \textit{<<История и классовое сознание>>} автора \textbf{Дьёрдя Лукача} и
работа \textit{<<Марксизм и философия>>} \textbf{Карла Корша}, вышедшие примерно
в одно время.

Д. Лукач внёс свой вклад в <<марксисткий ренессанс>> раскрытием философского
содержания <<Капитала>>. Причём марксизм был истолкован как совершенно особый
тип философии, в рамках которого общемировоззренческие категории органически
связаны с социально-историческими понятиями.
Лукачевское истолкование произвело весьма сильное впечатление на всех тех, кто видел
в Карле Марксе только политически ориентированного экономиста, в лучшем случае ---
широко мыслящего социолога.

Что касается К. Корша, то его роль в формировании неомарксизма была
значительно меньшая, чем лукачевская. Судя по тому, что писал о нём в 1950-е годы
французский философ феноменологического направления М. Мерло-Понти, Корш произвёл
впечатление на <<левую>> западно-европейскую интеллигенцию прежде всего своим
утверждением, согласно которому у Маркса наличествует две противоборствующие
мыслительные тенденции: первая --- философская, она же революционная; вторая ---
анти-философская, сциентистски-позитивистская. Первая, если верить Коршу,
доминировала в мышлении Маркса в периоды общественного подъёма, когда революция
казалась близкой. Вторая выдвигалась на передний план, подавляя первую, когда этот
подъём спадал и перспектива революции становилась всё более отдалённой

В период, когда первый <<марксистский ренессанс>> в буржуазной социально-философской
мысли достиг своей кульминации, появилась первая неомарксистская школа. Она возникла
на базе Франкфуртского института социальных исследований. Несмотря на то, что
институт был основан в 1923 году, только в 1929-1930-х годах, когда поста директора
этого института занял Макс Хоркхаймер, \textbf{<<Франкфуртсткая школа>>}
получила ярковыраженную неомарксистскую <<окраску>>.

Франкфуртская школа сыграла очень большую --- в некоторых отношениях решающую ---
роль в разработке и распространении неомарксисткого комплекса идей, благодаря
наличию института в качестве базы, журналу, ряда регулярно издаваемых теоретических
работ, единству общей концепции и так далее~\cite{neomarxism}.

В отличие от <<франкфуртской>> версии неомарксизма, иные его варианты развивающиеся
в других странах, таких как Франция, США, Англия, не были столь жестко
институционализированы и не получили такого систематического истолкования,
которое обеспечивала Франкфуртская школа.

В числе \textbf{французских неомарксистов} следует отметить \textbf{Анри Лефевра},
для которого неомарксизм был формой перехода от марксизма-ленинизма
к «левому» радикализму.

\textbf{Американский неомарксизм} уделяет серьезное внимание проблематике социологии
культуры. Одна из наиболее известных работ \textbf{Ч.~Р.~Миллса} --- \textit{<<Социологическое воображение>>},
которая представляет собой нечто вроде интеллектуального завещания автора
молодым социальным исследователям. В ней содержится не только критический анализ основных
направлений в социологической науке, а также позиции Миллса по основным проблемам
общественного развития, формируется ряд наиболее важных, по его мнению задач,
стоящих как пред обществом в целом, так и перед социальными науками и, в первую очередь,
перед социологией.

Что касается \textit{постмарксизма}, его, прежде  всего, связывают с
работой \textbf{Эрнесто Лакло} и \textbf{Шанталь Муфф}
\textit{<<Гегемония и Социалистическая традиция>>},
которая была издана в 1985 году. Для многих политических деятелей и
представителей критической теории эта работа является, пожалуй, ключевым
текстом постмарксизма, кроме того \textit{<<Гегемония>>} осмысляется и как подлинно
постмарскистский текст (\textit{Джерас 1987 г.}, \textit{Можелис 1988 г.}).
Именно в этой работе видно желание признать важность марксистской теории, что раскрывается
в задаче сформировать левый радикальный дискурс <<после>> исчезновения
марксизма с исторической сцены~\cite{key_thinkers}.
