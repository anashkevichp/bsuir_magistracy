\section*{ЗАКЛЮЧЕНИЕ}
\addcontentsline{toc}{section}{Заключение}

Несоответствие современным условиям ряда ключевых положений и внутренняя
противоречивость марксизма стимулировали попытки его модернизации.
Одной из наиболее известных модификаций марксизма в двадцатом веке был так называемый
неомарксизм, организационно оформившийся в 1930-х годах на базе Франкфуртской
школы социальных исследований.

Позднее, после событий <<красного мая>> 1968 года, на базе марксистских вглядов
возник постмарксизм, разрушивший понятия <<западный>> и <<восточный>> неомарксизм.

В целом как неомарксизм, так и постмарксизм отражали поиск марксистскими и
марксистски-ориентированными мыслителями <<третьего пути>>, свободного как от буржуазности,
отчуждения и манипуляций массовым сознанием, так и от тоталитаризма
с присущими ему бюрокра­тей и государственной идеологии.

\pagebreak
