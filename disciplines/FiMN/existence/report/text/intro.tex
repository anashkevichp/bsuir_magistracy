\section*{ВВЕДЕНИЕ}
\addcontentsline{toc}{section}{Введение}

Экзистенциализм --- это направление в современной западной философии,
возникшее в ХХ веке как попытка создания нового, по сравнению с существующим,
мировоззрения, отвечающего взглядам современного человека~\cite{solecity_exist}.
Главным предметом философии экзистенциализма является человек.
В рамках данной философии он выступает одновременно и как субъект, и как объект.

Основателями экзистенциализма считаются Карл Ясперс и Мартин Хайдеггер.
Выделяют две волны философов-представителей данного учения:
первая возникла в Германии после первой мировой войны;
вторая --- во Франции после второй мировой войны.
Кроме этого, различают два направления экзистенциализма: религиозный и атеистический.

Одним из главных представителей второй волны атеистического экзистенциализма
считается Жан-Поль Сартр. Наиболее известным его произведением является роман <<Тошнота>>,
в котором описываются поиски главного героя ответов на вопросы бытия.
В 1964 году Ж. П. Сартр был удостоен Нобелевской премии по литературе,
от которой он отказался, мотивировав свой отказ нежеланием ограничивать свою свободу,
будучи обязанным какому-либо социальному институту~\cite{sartr_nobel}.

В данной работе кратко рассматриваются основные положения философии атеистического
экзистенциализма. В качестве основного источника информации о данном учении используется
статья Ж. П. Сартра <<Экзистенциализм --- это гуманизм>>~\cite{sartr_exist_human}.
