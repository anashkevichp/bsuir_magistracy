\subsection{Bug-Injection Scenarios} \label{sec:bug_cases}
Because we model non-preemptible Tree RCU, each CPU runs exactly one RCU task
as a separate thread.
Upon completion, each task increments a global counter \co{thread_cnt},
enabling the parent thread to verify the completion of all RCU tasks
using a statement \co{__CPROVER_assume(thread_cnt == 2)}.
The base case uses the example in Figure~\ref{fig:verify_rcu_gp}, including
its assertion \co{assert(r2 == 0 || r1 == 1)}.
This assertion does not hold when
RCU's fundamental safety guarantee is violated:
read-side critical sections cannot span grace
periods~\cite{DesnoyersTPDS12UserRCU}.
%
We also verify a \emph{weak form} of liveness by inserting an \co{assert(0)} 
after the \co{__CPROVER_assume(thread_cnt == 2)} statement.
This assertion cannot hold, and so it will be violated if at least one grace period completes.
Such a ``verification failure" is in fact the expected behavior for a correct RCU implementation. 
On the other hand, if the assertion is not violated, grace periods never complete, which indicates a liveness bug.

To validate our verification, we also run CBMC with the bug-injection scenarios 
described below,\footnote{Source code is available: \url{http://lxr.free-electrons.com/source/kernel/rcu/?v=4.3}}
which are simplified versions of bugs encountered in actual practice.
Bugs~2--6 are liveness checks and thus use the
aforementioned \co{assert(0)}, and the remaining scenarios are
safety checks which thus use the base-case assertion in
Figure~\ref{fig:verify_rcu_gp}.
% \comment{Lihao: we might explain each bug scenario in greater detail 
% assuming the readers aren't familiar with the code.}

\paragraph*{Bug 1}
This bug-injection scenario makes
the RCU update-side primitive \co{synchronize_rcu()} return immediately 
(line 523 in \co{tree_plugin.h}).
With this injected bug, updaters never wait for readers, which should
result in a safety violation, thus preventing
Figure~\ref{fig:verify_rcu_gp}'s assertion from holding.

\paragraph*{Bug 2}
The key idea behind this bug-injection scenario is to prevent individual
CPUs from realizing that quiescent states are needed, thus preventing
them from recording quiescent states. As a result, it prevents grace periods
from completing.
Specifically, in function \co{rcu_gp_init()}, for each \co{rcu_node}
structure in the combining tree,
we set the field \co{rnp->qsmask} to 0 instead of \co{rnp->qsmaskinit} (line 1889 in \co{tree.c}). 
Then when \co{rcu_process_callbacks()} is called, \co{rcu_check_quiescent_state()} will invoke
\co{__note_gp_changes()} that sets \co{rdp->qs_pending} to 0. Thus,  
\co{rcu_check_quiescent_state()} will return without calling \co{rcu_report_qs_rdp()},
preventing grace periods from completing.
This liveness violation should fail to trigger a violation of the end-of-execution
\co{assert(0)}.

\paragraph*{Bug 3}
This bug-injection scenario is a variation of Bug 2, in
which each CPU remains aware that quiescent states are required, but
incorrectly believes that it has already reported a quiescent state
for the current grace period.
To accomplish this,
in \co{__note_gp_changes()}, we clear \co{rnp->qsmask} by adding a statement
\co{rnp->qsmask &= ~rdp->grpmask;} in the last \co{if} code block (line 1739 in \co{tree.c}). 
Then function \co{rcu_report_qs_rnp()} never walks up the \co{rcu_node}
tree, resulting in a liveness violation as in Bug~2.

\paragraph*{Bug 4}
This bug-injection scenario is an alternative code change that gets the
same effect as does Bug~2.
For this alternative,
in \co{__note_gp_changes()}, we set the \co{rdp->qs_pending} field to 0 directly 
(line 1749 in \co{tree.c}). This is a variant of Bug 2 and thus also
a liveness violation.

\paragraph*{Bug 5}
In this bug-injection scenario, CPUs remain aware of the need for
quiescent states.
However, CPUs are prevented from recording their
quiescent states, thus preventing grace periods from ever completing.
To accomplish this, we modify
function \co{rcu_sched_qs()} to return immediately (line 246 in \co{tree.c}),
so that quiescent states are not recorded.
Grace periods therefore never complete, which constitutes a liveness
violation similar to Bug~2.

\paragraph*{Bug 6}
In this bug-injection scenario, CPUs are aware of the need for quiescent
states, and they also record them locally.
However, they are prevented from reporting them up the \co{rcu_node}
tree, which again prevents grace periods from ever completing.
This bug modifies
function \co{rcu_report_qs_rnp()} to return immediately (line 2227 in \co{tree.c}). 
This prevents RCU from walking up the \co{rcu_node} tree, thus preventing grace
periods from ending.
This is again a liveness violation similar to Bug~2.

\paragraph*{Bug 7}
Where Bug~6 prevents quiescent states from being reported up the
\co{rcu_node} tree, this bug-injection scenario causes quiescent
states to be reported up the tree prematurely, before all the
CPUs covered by a given subtree have all reported quiescent states.
To this end,
in \co{rcu_report_qs_rnp()}, we remove the \co{if}-block checking for
\co{rnp->qsmask != 0 || rcu_preempt_blocked_readers_cgp(rnp)} (line 2251 in \co{tree.c}). 
Then the tree-walking process will not stop until it reaches the root, resulting in 
too-short grace periods.
This is therefore a safety violation similar to Bug~1.

\noindent Bugs~2 and~3 would result in a too-short grace period given quiescent-state forcing, 
but such forcing falls outside the scope of this paper.
