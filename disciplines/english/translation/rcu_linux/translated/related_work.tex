\section{Related Work}
%Formal verification has been applied to various aspects of RCU design 
%and implementation. 
%
McKenney applied the SPIN model checker to verify RCU's \co{NO_HZ_FULL_SYSIDLE}
functionality \cite{VerificationChallenges},
and interactions between dyntick-idle and 
non-maskable interrupts~\cite{ValDyntickNMI}. 
%
Desnoyers et al.~\cite{DesnoyersOSR13} propose a virtual architecture 
to model out-of-order memory accesses and instruction scheduling.
User-level RCU \cite{DesnoyersTPDS12UserRCU} is modeled and verified 
in the proposed architecture using the SPIN model checker. 

These efforts require an error-prone translation from C to SPIN's 
modeling language, and therefore are not appropriate for regression testing. 
By contrast, our work constructs an RCU model directly from its 
source code from the Linux kernel, and verifies it using automated 
verification tool. 

Alglave et al.~\cite{AlglaveCAV13} introduce a symbolic encoding 
for verifying concurrent software over a range of memory models 
including SC, TSO and PSO.
They implement the encoding in the CBMC bounded model checker and 
use the tool to verify \co{rcu_assign_pointer()} and \co{rcu_dereference()}.

McKenney used CBMC to verify Tiny
RCU~\cite{VerificationChallenges}, a trivial Linux-kernel
RCU implementation for uni-core systems.

Groce et al.~\cite{GroceASE15RCU} introduce a falsification-driven
verification methodology that is based on a variation of mutation 
testing. By using CBMC, they were able to find two holes in 
\co{rcutorture}--RCU's stress testing suite, one of which was hiding 
a real bug in Tiny RCU.
Further work on real hardware identified two more
\co{rcutorture} holes, one of which was hiding a real bug in Tasks
RCU~\cite{JonathanCorbet2014RCU-tasks} and the other of which was hiding
a minor performance bug in Tree RCU.
% \comment{Lihao: shall we add a reference to RCU-tasks. I found this one:
% https://lwn.net/Articles/607117/}
% Paul: Done!

In this work, we use CBMC to verify the implementation of Linux-kernel
Tree RCU for multi-core systems, which is more complex
and sophisticated, over SC, TSO, and PSO.

Gotsman et al.~\cite{YangESOP13RCU} use a extended concurrent separation logic 
%extended with temporal operators 
to formalise the concept 
of grace period and prove an abstract implementation of RCU over SC.
%
Tassarotti et al.~\cite{DreyerPLDI15RCU} use GPS, a recently 
developed program logic for the C/C++11 memory model, to carry out 
a formal proof of a simple implementation of user-level 
RCU for a singly-linked list assuming ``release-acquire'' semantics, 
which is weaker than SC but stronger than memory models used by
real-world RCU implementations.
%
These formal proofs were performed manually on simple implementations 
of RCU. By contrast, our work applies an automated verification tool with a 
test harness to verify the grace-period property of a real-world
implementation of RCU over SC, TSO, and PSO.

Formal verification has started to make its way into real-world practice 
of verifying large non-trivial code bases. Cal\-ca\-gno et al.~\cite{CalcagnoNASA15} 
describe integrating a static-analysis tool
into Facebook's software development cycle.
We believe that our work is an 
important step towards integration of verification into Linux-kernel RCU's 
regression test suite.
