\subsection{Getting CBMC to work on Tree RCU} \label{sec:cbmc_on_rcu}

We have found that getting CBMC to work on our RCU model is non-trivial due
to Tree RCU's complexity combined with CBMC's bit-precise verification.  In
fact, early attempts resulted in SAT formulas that were so large that CBMC
ran out of memory.  After the optimizations described in the remainder of
this section, the largest formula contained around 90 million variables
and 450 million clauses, which enabled CBMC to run to completion.
%
% \comment{Paul: Do we have formula sizes from before this effort for
% purposes of comparison? Lihao: we do, but with different versions of the code. 
% For example, one version has 168 million variables and 738 million clauses. 
% Not sure which one to use, similar to the situation in the next comment.}

First, instead of placing the scheduling-clock interrupt in its own thread,
we invoke functions \co{rcu_note_context_switch()} and
\co{rcu_process_callbacks()} directly, as described in
Section~\ref{sec:model_irq}.  Also, we invoke
\co{__note_gp_changes()} from \co{rcu_gp_init()} to notify each CPU of a new
grace period, instead of invoking \co{rcu_process_callbacks()}.
%
%These changes reduced CBMC's common-case runtime from more than 24 hours to less than 10 hours.
% \comment{Paul: Can we provide this sort of blow-by-blow result for all steps?
% If not, should we should remove this one?
% Or give a qualitative comparision: ``Before applying this optimization,
% CBMC ran out of memory before completing the formula.'' Lihao: let's do the latter.}

Second, the support for linked lists in CBMC version 5.4 is limited,
resulting in unreachable code in CBMC's symbolic execution. Thus, we
stubbed all the list-related code in our RCU model, including those for
callback handling.

Third, CBMC's structure-pointer and array encodings result in large formulas
and long formula-generation times.  Our focus on the RCU-sched flavor
allowed us to eliminate RCU-BH's data structures and trivialize the
\co{for_each_rcu_flavor()} flavor-traversal loops.  Our focus on small
numbers of CPUs meant that RCU-sched's \co{rcu_node} tree contained only a
root node, so we also trivialized the \co{rcu_for_each_node_breadth_first()}
loops traversing this tree.

Fourth, CBMC unwinds each loop to the depth specified in its command line
option \co{--unwind}, even when the actual loop depth is smaller.  This
unnecessarily increases formula size, especially for loops containing
intricate RCU code. Since loops in our model can be decided at compile time, 
we therefore used the command line option \co{--unwindset} to specify 
unwinding depths for each individual loop.

Finally, since our test harness only requires one \co{rcu_node} structure
and two \co{rcu_data} structures, we can use 32-bit encodings for \co{int},
\co{long}, and pointers by using the command line option \co{--ILP32}.  This
reduces CBMC's formula size by half compared to the 64-bit default.
