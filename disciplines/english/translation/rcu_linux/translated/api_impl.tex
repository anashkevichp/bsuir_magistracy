%\subsection{Read-Side Primitives} \label{sec:read_api_impl}
\subsection{Read/Write-Side Primitives} \label{sec:api_impl}
% Change in recent kernels.
In a non-preemptible kernel, any region of kernel code that does not voluntarily
block is implicitly an RCU read-side critical section.
Therefore, the implementations of \co{rcu_read_lock()} and
\co{rcu_read_unlock()} need do nothing at all, and in fact
in production kernel builds that do not have debugging enabled,
these two primitives have absolutely no effect on code generation.

%\comment{Lihao: comment out the following preemptible RCU contents if we need space.}
%In a preemptible kernel, \co{rcu_read_lock()} increments 
%the variable \co{rcu_read_lock_nesting} in the per-task
%\co{task_struct} structure, and
%\co{rcu_read_unlock()} decrements this same variable.
%If the result is zero 
%(outermost \co{rcu_read_unlock()}) and the per-CPU variable 
%\co{rcu_read_unlock_special} in the task structure is non-zero, it will also 
%invoke \co{rcu_read_unlock_special()} to clean up after a context-switch 
%in an RCU read-side critical section and other special cases. 
% Lihao: http://lxr.free-electrons.com/source/kernel/rcu/update.c#L143

%\subsection{Update-Side Primitives} \label{sec:update_api_impl}
%In a non-preemptible kernel, \co{synchronize_rcu()} simply calls
% \co{synchronize_sched()}, which is defined in \co{kernel/rcu/tree.c}. 
% If \co{synchronize_sched()} sees that there is only one online CPU,
% then, just as with Tiny RCU, the grace period immediately completes.
%
In the common case where there are multiple CPUs running,
% if grace periods should be expedited, it will
% call \co{synchronize_sched_expedited()} to force the grace period to
% end quickly.
% Lihao: http://lxr.free-electrons.com/source/kernel/rcu/tree.c#L3170
%
% If no grace period expediting is needed, it instead calls
the update-side primitive \co{synchronize_rcu()} calls
\co{wait_rcu_gp()}, which is an internal function that uses
a callback mechanism to invoke \co{wakeme_after_rcu()}
at the end of some later grace period.
As its name suggests, \co{wakeme_after_rcu()} function wakes up
\co{wait_rcu_gp()}, which returns, in turn allowing
\co{synchronize_rcu()} to return control to its caller.

%\comment{Lihao: comment out the following preemptible RCU contents if we need space.}
%In a preemptible kernel, \co{synchronize_rcu()} is implemented in
%\co{kernel/rcu/tree_plugin.h}. It first checks whether the variable 
%\co{rcu_scheduler_active} is zero. If so, the system is so early in boot
%that there is only one non-preemptible task, again meaning that grace
%periods complete instantaneously, allowing an immediate return.
%Otherwise, if the grace period should be expedited,
%\co{synchronize_rcu_expedited()} is invoked.
%Otherwise, it passes \co{call_rcu()} to \co{wait_rcu_gp()}, which
%registers callback \co{wakeme_after_rcu()}, similar to
%the non-preemptible kernel discussed above. 
%%\comment{Lihao: the source code comments state that \co{rcu_scheduler_active = 0} 
%%allows RCU to optimize \co{synchronize_sched()} to a simple \co{barrier()}. 
%%Where is the code that does this?}
%%\comment{Paul: The comment is incorrect.
%%The \co{synchronize_sched()} function instead checks the number of
%%online CPUs.
%%I have queued a patch with your
%%Reported-by changing the comment's \co{synchronize_sched()} to
%%\co{synchronize_rcu()}.}
%
%RCU's callback handling and grace period detection are explained in Sections 
%\ref{sec:rcu_data} and \ref{sec:grace_period}, respectively.

% Lihao: understand how call_rcu_sched works and understand the differences from 
% the Tiny RCU version which only calls the kernel function cond_resched()
%
% Lihao: In a preemptible kernel, the implementation of \co{synchronize_rcu()}
% http://lxr.free-electrons.com/source/kernel/rcu/tree_plugin.h#L539 and
% understand the differences between preemptible and non-preemptible versions

