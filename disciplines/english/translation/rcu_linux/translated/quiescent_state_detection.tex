\subsection{Quiescent State Detection} \label{sec:quiescent_state}
RCU has to wait until all pre-existing read-side critical sections have
finished before it can safely allow a grace period to end.
The performance and scalability of RCU rely on its ability to efficiently
detect quiescent states and determine whether the set of quiescent states
detected thus far allows the grace period to end.
If each CPU (or, in the case of preemptible RCU, each task)
has passed through a quiescent state, a grace period has elapsed. 

% Lihao: include this in PhD thesis; also look for 'preemptible RCU contents'
%Different flavors of RCU use different sets of quiescent states.
The non-preemptible RCU-sched flavor's quiescent states
apply to CPUs, and are user-space execution, context switch, idle, and 
offline state.
%
%RCU-bh's quiescent states are those of RCU-sched plus any execution 
%in which bottom-half (AKA softirq) is enabled, along with transitions
%from one softirq handler to another.
%%
%RCU-preempt's quiescent states are any execution outside of an
%RCU read-side critical sections.
%
Therefore, RCU-sched %and RCU-bh need only
only needs to track tasks and interrupt handlers that are actually running because
blocked and preempted tasks are always in quiescent states. Thus, RCU-sched %and RCU-bh 
needs only track CPU states.
%By contrast, RCU-preempt must track tasks states. In this section, 
%we focus on the quiescent-state detection of RCU-sched in a non-preemptible kernel.
%\comment{Lihao: comment out the following preemptible RCU contents if we need space.}
%However, there can be a great many tasks, and scanning all of them could
%result in excessive per-grace-period overheads.
%However, if a task has been preempted or has blocked outside of an RCU
%read-side critical section, its state need not be considered.
%Therefore, a given grace period need only consider tasks that have been
%preempted (or, in real-time variants of the Linux kernel, blocked)
%within an RCU read-side critical section that began before the current
%grace period did.
%These are exactly the tasks that are tracked by the \co{rcu_node} structure's
%\co{->blkd_tasks} list and \co{->gp_tasks} pointer, as discussed in
%Section~\ref{sec:rcu_node}.

\subsubsection{Scheduling-Clock Interrupt} \label{sec:timer_interrupt}
The \co{rcu_check_callbacks()} is invoked from the sched\-ul\-ing-clock interrupt
handler, which allows RCU to periodically check whether a given busy CPU
is in the user-mode or idle-loop quiescent states.
If the CPU is in one of these quiescent states, \co{rcu_check_callbacks()}
invokes \co{rcu_sched_qs()}, %and \co{rcu_bh_qs()}, 
which sets the per-CPU \co{rcu_sched_data.passed_quiesce} %and \co{rcu_bh_data.passed_quiesce}
fields to 1. %, respectively.
%In addition, when the scheduling-clock interrupt happens outside of softirq context,
%that is, outside of the softirq handler and also outside of any
%bottom-half-disable mode, then the CPU is in an RCU-bh quiescent state,
%in which case \co{rcu_check_callbacks()} will invoke \co{rcu_bh_qs()}
%to inform RCU of this quiescent state.

%\comment{Lihao: comment out the following preemptible RCU contents if we need space.}
%When RCU-preempt is present,
%\co{rcu_check_callbacks()} invokes
%\co{rcu_preempt_check_callbacks}, which
%checks for RCU-preempt quiescent states, which are in effect whenever
%the per-task \co{->rcu_read_lock_nesting} field is equal to 0.
%This condition causes \co{rcu_preempt_check_callbacks} to invoke
%\co{rcu_preempt_qs()}, which in turn records
%quiescent state for by setting the per-CPU \co{rcu_data_p->passed_quiesce} 
%field to true.
%
The \co{rcu_check_callbacks()} function invokes \co{rcu_pending()}
to determine whether a recent event or current condition means that
RCU requires attention from this CPU.
% so the next outermost \co{rcu_read_unlock()} will announce a quiescent state.
% http://lxr.free-electrons.com/source/kernel/rcu/tree_plugin.h#L485
If so, \co{rcu_check_callbacks()} invokes \co{raise_softirq()}, %with a \co{RCU_SOFTIRQ} argument, 
which will cause \co{rcu_process_callbacks()} to be invoked once the CPU 
reaches a state where it is safe to do so (roughly speaking, once the CPU 
has interrupts, preemption, and bottom halves enabled). This function is 
discussed in detail in Section \ref{sec:grace_period}.

% Paul: Yes, the order is different than the code, but it seems to flow better
% this way.  Perhaps I should re-order the code.  Except that I don't touch
% that particular piece of code without an extremely good reason.  ;-)

% Finally, if the scheduling-clock interrupt is raised in the user mode, we perform
% a context switch.
% Lihao: voluntary context switch?
% Paul: This is for Tasks RCU, which is probably out of scope.  That said,
% This function is not -performing- a voluntary context switch, but rather
% checking to see if a voluntary context switch has been performed.  In
% the context of Tasks RCU, executing in user mode is equivalent to having
% performed a voluntary context switch -- either way, whatever was happening
% in the kernel beforehand is now well and truly done.
% Lihao: understand the differences from the tiny RCU version and how the tiny RCU one 
% is related to cond_resched in synchronize_sched
% Lihao: understand how it is related to wait_rcu_gp(call_rcu_sched) in synchronize_sched()

\subsubsection{Context-Switch Handling} \label{sec:context_switch}
The context-switch quiescent state is recorded by invoking
\co{rcu_note_context_switch()} from \co{__schedule()} (and, for the
benefit of virtualization, also from \co{rcu_virt_note_context_switch()}).
% http://lxr.free-electrons.com/source/kernel/sched/core.c#L3057
%
The \co{rcu_note_context_switch()} function invokes \co{rcu_sched_qs()}
to inform RCU of the context switch, which is a quiescent state of the CPU.
%Note that although quiescent states of RCU-bh include those of RCU-sched,
%\co{rcu_note_context_switch()} does not invoke \co{rcu_bh_qs()}.
%This could in theory starve RCU-bh grace periods if a given CPU spent all
%its time in the kernel in bottom-half-disabled regions, without any
%calls to \co{schedule()}.
%No part of the kernel currently does this, but should this pattern arise,
%RCU-bh's quiescent-state recording strategy will need to be revisited.

%\comment{Lihao: comment out the following preemptible RCU contents if we need space.}
%It also invokes \co{rcu_preempt_note_context_switch()} to add the current
%task to the \co{->blkd_tasks} list of the CPU's leaf \co{rcu_node}
%structure for context switches that occur within an RCU-preempt read-side
%critical section.
%To prevent this task from being re-added while within its current
%RCU-preempt read-side critical section,
%the first \co{rcu_preempt_note_context_switch()} sets the
%\co{->rcu_read_unlock_special.b.blocked} field in the task structure.
%
%However, if current task has already reported an RCU-preempt
%quiescent state for the current grace period, and if at least one
%other task is blocking that grace period on this \co{rcu_node}
%structure,
%the task should be added to the head of the \co{->blkd_tasks} list
%in order to avoid blocking that grace period.
%In this case, the \co{->gp_tasks} field
%will be non-\co{NULL} and the current CPU's bit will already be cleared
%from the \co{->qsmask} field.
%In all other cases, the task should be added to the tail of the
%\co{->blkd_tasks} list.
%If the task is blocking the current RCU-preempt grace period and
%\co{->gp_tasks} is \co{NULL}, then this is the first task on this
%leaf \co{rcu_node} structure to block the this grace period, and
%therefore \co{->gp_tasks} is set to reference the current task.
%This approach allows RCU to easily identify which tasks are blocking
%the current grace period.
%
%The \co{rcu_preempt_note_context_switch()} function also invokes
%\co{rcu_preempt_qs()} to note a quiescent state for the current CPU.
%Nevertheless, any tasks queued on the \co{->gp_tasks} segment of
%\co{->blkd_tasks} will continue to block the grace period.
%
%All of the tasks on the \co{->blkd_tasks} list dequeue themselves
%during the outermost \co{rcu_read_unlock()}.
%This of course introduces a race condition where a task is preempted
%while executing its outermost
%\co{rcu_read_unlock()}~\cite{PaulEMcKenney2011RCU3.0trainwreck}.
%\comment{Paul: This citation isn't all that important, so feel free to remove.}
%This race is detected by having \co{rcu_read_unlock()} set the \co{task_struct}
%structure's \co{->rcu_read_lock_nesting} field to a negative value.
%When \co{rcu_preempt_note_context_switch()} sees this negative value,
%it invokes \co{rcu_read_unlock_special()} to complete the dequeuing
%of the current task from the \co{->blkd_tasks} list.
%Interrupt disabling prevents further destructive races.
%% Lihao: if show code, add the following text
%% Recall that \co{rcu_read_unlock()} sets \co{rcu_read_lock_nesting} of its  
%%\co{task_struct} structure to \co{INT_MIN} before invoking 
%%\co{rcu_read_unlock_special()} 
%%(and to 0 after \co{rcu_read_unlock_special()} returns)
%% If the {task_struct}'s \co{rcu_read_unlock_special.s} is not equal to 0,
%% we have work left to do for \co{rcu_read_unlock_special}, which is then invoked.
%% http://lxr.free-electrons.com/source/kernel/rcu/tree_plugin.h#L267 
%% Lihao: may need to explain the code of rcu_read_unlock_special() 
%% Paul: I did a minimal explanation, which can be expanded if necessary.
