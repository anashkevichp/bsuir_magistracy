\subsection{Обнаружение устойчивых состояний} \label{sec:quiescent_state}
Механизм RCU должен ожидать до тех пор,
пока все потоки не выйдут из своих критических секций чтения перед тем,
как можно будет завершить grace-период.
Производительность и масштабируемость RCU
основывается на его способности быстро обнаруживать
устойчивые состояния вычислительных ядер и определять
момент, когда их набралось достаточно, чтобы завершить
grace-период.
Если каждое ядро (или, в случае preemptible-RCU, каждый поток)
прошел через устойчивое состояние, то можно считать,
что grace-период закончился.

В случае использования non-preemptible RCU-sched вида RCU,
устойчивыми состояниями считаются следующие состояния вычислительных ядер:
выполнение инструкций пользовательского пространства,
переключение контекста, режим ожидания и offline-режим.
RCU-sched отслеживает лишь потоки и векторы прерываний,
которые выполняются в данный момент,
поскольку заблокированные и прерванные потоки всегда находятся в
устойчивых состояниях.
Таким образом, RCU-sched достаточно отслеживать состояния вычислительных ядер.

\subsubsection{Таймер прерываний} \label{sec:timer_interrupt}
Функция \co{rcu_check_callbacks()} вызывается из обработчика таймера прерываний,
позволяющего RCU периодически проверять, находится ли данное вычислительное
ядро в пользовательском режиме или в одном из устойчивых состояний.
Если ядро находится в одном из этих состояний,
\co{rcu_check_callbacks()} вызывает \co{rcu_sched_qs()},
который изменяет значение поля \co{rcu_sched_data.passed_quiesce} для
каждого ядра.
Функция \co{rcu_check_callbacks()} вызывает \co{rcu_pending()}
для того, чтобы проверить, является ли последнее событие или данное условие
признаком внимания к данному ядру со стороны RCU.
Если да, то \co{rcu_check_callbacks()} вызывает функцию \co{raise_softirq()},
которая приводит к тому, что \co{rcu_process_callbacks()} будет вызвана,
как только ядро достигнет безопасного состояния
(грубо говоря, когда на ядре будут включены прерывания,
preemption и bottom halves).
Эта функция подробно рассматривается в разделе \ref{sec:grace_period}.

\subsubsection{Управление переключениями контекста} \label{sec:context_switch}
Устойчивые состояния, связанные с переключениями контекста,
учитываются путем вызова функции \co{rcu_note_context_switch()} из
\co{__schedule()}
(и, для поддержки виртуализации,
из \co{rcu_virt_note_context_switch()}).
Функция \co{rcu_note_context_switch()} вызывает \co{rcu_sched_qs()}
для оповещения RCU о переключении контекста, которое является устойчивым
состоянием вычислительного ядро.
