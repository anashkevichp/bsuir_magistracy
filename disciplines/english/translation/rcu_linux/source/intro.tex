\section{Introduction}
The Linux operating system kernel~\cite{LinuxKernel} is widely used 
in a variety of computing platforms, including servers, safety-critical 
embedded systems, household appliances, and mobile devices such as 
smartphones. Over the past 25 years, many technologies 
have been added to the Linux kernel, one example being Read-Copy 
Update (RCU)~\cite{McKenneyRCU98}.

RCU is a synchronization mechanism that can be used to replace reader-writer 
locks in read-mostly scenarios. It allows low-overhead readers
to run concurrently with updaters. Production-quality RCU implementations 
for multi-core systems must provide excellent scalability, 
high throughput, low latency, modest memory footprint, excellent energy 
efficiency, and reliable response to CPU hotplug operations.
The implementation must therefore avoid cache misses, lock contention, 
frequent updates to shared variables, and excessive 
use of atomic read-modify-write and memory-barrier instructions.
Finally, the implementation must cope with the extremely diverse
workloads and platforms of Linux~\cite{McKenneyOSR08}.

RCU is now widely used in the Linux-kernel networking, device-driver, and
file-storage
subsystems~\cite{McKenneyOSR08,McKenneyRCUsageReport13}.
%\url{http://www2.rdrop.com/~paulmck/RCU/linuxusage/linux-4.3.sum}}
To date, there are at least 75 million Linux servers \cite{LinuxServerNum13} 
and 1.4 billion Android devices~\cite{AndroidNum15},
which means that a ``million-year'' bug
can occur several times per day across the installed base.
Stringent validation of RCU's complex implementation is thus critically
important.

Most validation efforts for concurrent software rely on testing, but
unfortunately there is no cost-effective test strategy that can cover
all corner cases. %and detect all bugs.
Worse still, some of errors that testing does detect might be difficult to 
reproduce, diagnose, and repair. The concurrent nature of RCU and the sheer 
size of the search space suggest use of formal verification, particularly
model checking~\cite{BurchInfComput92}.

Formal verification has already been applied to some aspects of RCU design, 
including Tiny RCU~\cite{VerificationChallenges}, userspace RCU~\cite{DesnoyersOSR13}, 
sysidle~\cite{VerificationChallenges}, and interactions between
dyntick-idle and non-maskable interrupts (NMIs)~\cite{ValDyntickNMI}.
% Lihao: correct citations? Paul: Look good. For a more recent NMI/dyntick example: 
% \url{https://kernel.googlesource.com/pub/scm/linux/kernel/git/paulmck/linux-rcu/+/andy.2014.11.21a}.}
%
But these efforts either validate trivial single-CPU RCU implementations
in~C (Tiny RCU), or use special-purpose languages such as
Promela~\cite{HolzmannTSE97SPIN}.  Although special-purpose modeling
languages do have advantages, a major disadvantage in the context of the
Linux kernel is the difficult and error-prone translation from source code.
%
Other researchers have applied manual proofs in formal logics to simple 
RCU implementations~\cite{YangESOP13RCU,DreyerPLDI15RCU}. These proofs are %of course
quite admirable, but require even more manual work, in addition
to the translation effort.

Worse yet, Linux kernel releases are only about 60 days apart, and RCU
changes with each release.  Thus, any manual work must be replicated about
six times a year so that mechanical and manual models or proofs remain
synchronized with the Linux-kernel RCU implementation.
%
Therefore, if formal verification is to be part of Linux-kernel RCU's 
regression suite, the verification methods must be scalable and automated. 
% Paul: I say "must be scalable and automated" with the power invested in
% me in my role as Linux-kernel RCU maintainer.  ;-)
%, especially if it is to be applied to the full Linux kernel as opposed to 
% the tiny subset of the kernel that is RCU.
%
To this end, this paper describes how to build a model directly from the
Linux kernel source code, and use the C~Bounded Model Checker
(CBMC)~\cite{KroeningTACAS04CBMC} to verify RCU's safety and liveness
properties.
%
To the best of our knowledge, this is the first automatic
verification of a significant part of the Linux-kernel RCU source code.
