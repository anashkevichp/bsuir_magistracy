\section[The growth of mobile]{THE GROWTH OF MOBILE}

For many employees, desktop computing is a thing of the past.
Or, at the very least, advances in mobile technology over the past few years
have made it easier than ever to work remotely.
What once required a computer can now often be done on a smartphone.

The number of hours that people spend online today continues
to grow every year. These hours have shifted from desktop computers
to mobile phones to a range of other devices. In fact, the amount
of time spent on the mobile web increased 575\% in around 3 years.

It is not just the amount of time that people spend online that matters.
People now have more devices to choose from when they want to go online.
As the adoption of mobile devices grows, the amount of time that
people spend on them has grown with it. In fact, the amount of time
spent on mobile internet has now overtaken the amount of time spent online on a desktop.

These trends do not happen in a vacuum. Mobile internet did not shift from
the development BBM to free Netflix streaming overnight. A series of events
had to take place in order for mobile internet to be feasible at this level.

First, mobile devices needed to be more accessible. American carriers took care
of that by leasing these phones alongside service contracts. This had benefits
for everyone because consumers no longer had to shell out hundreds of dollars
up front and carriers could make a bundle by locking those consumers into contracts.

Second, the availability of mobile internet had to improve. Even the youngest
adopters can remember a time when mobile internet was good for chatting
and email but still couldn’t beat wifi. Carriers knew that the cost of
selling more data would be investment in better data services including
better technology and more towers.

Third, mobile devices and data had to be useful in a way that was not
currently available on the PC. Thus, the mobile app was born. Today, mobile
apps are changing the game completely and users are abandoning their
mobile browsers for apps. Google reported that its Android market
received 50 billion app downloads in 2013 alone.
Some reports suggest that this will grow to 268 billion downloads by 2017.

Apps are not just a new trend. Nearly 80\% of mobile hours are spent
using apps and total app usage grew 76\% in 2014 alone. This is because apps
offer a unique value proposition that merges convenience and sleek design.
For many customers, this value proposition is worth enough to drag
even the most dedicated PC user onto their phone or tablet.
\textbf{[https://www.cleverism.com/mobile-usage-implications-for-growing-your-business/]}

Worldwide combined shipments of devices (PCs, tablets, ultramobiles
and mobile phones) are expected to reach 2.4 billion units in 2016, a 1.9 percent
increase from 2015, according to Gartner, Inc. End-user spending
in constant U.S. dollars is expected to decline 0.5 percent for the first time.

\begin{table} [h!]
  \caption{
    Worldwide devices shipments by device type, 2015-2018 \\
    \hspace{29.5mm} (millions of units)
  }\label{tbl:temp}
  \begin{tabular}{| m{9.8cm} | c | c | c | c |}
    \hline
    Device Type & 2015 & 2016 & 2017 & 2018 \\
    \hline

    Traditional PCs (Desk-Based and Notebook) & 246 & 232 & 226 & 219 \\
    \hline

    Ultramobiles (Premium) & 45 & 55 & 74 & 92 \\
    \hline

    PC Market & 290 & 287 & 299 & 312 \\
    \hline

    Ultramobiles (Basic and Utility) & 196 & 195 & 196 & 198 \\
    \hline

    Computing Devices Market & 486 & 482 & 495 & 510 \\
    \hline

    Mobile Phones & 1 910 & 1 959 & 1 983 & 2 034 \\
    \hline

    \textbf{Total} & \textbf{2396} & \textbf{2441} & \textbf{2 478} & \textbf{2 545} \\
    \hline

  \end{tabular}
\end{table}

The device market in 2016 will continue to be impacted by country-level economic
conditions. <<It’s clear that vendors can no longer market their products with
the mind of only targeting the mature and emerging markets,>> said Ranjit Atwal,
research director at Gartner. <<Driven by economic variations the market is
splitting into four categories: economically challenged mature markets,
economically stable mature markets and the same for emerging markets.
Russia and Brazil will fall into the category of economically challenged
emerging markets while India will be stable, and Japan will belong
to the economically challenged mature market.>>
\textbf{[http://www.gartner.com/newsroom/id/3187134]}






\pagebreak