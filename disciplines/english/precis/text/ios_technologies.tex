\section[iOS app development technologies]
{IOS APP DEVELOPMENT TECHNOLOGIES}

iOS is the operating system that runs on iPad, iPhone, and iPod touch devices.
The operating system manages the device hardware and provides the technologies
required to implement native apps. The operating system also ships with
various system apps, such as Phone, Mail, and Safari, that provide standard
system services to the user.


\subsection{Programming languages}

The are two languages which can be used to create iOS applications.
Consider each of them.

Objective-C is a general-purpose, object-oriented programming language that adds
Smalltalk-style messaging to the C programming language. This is the main
programming language used by Apple for the OS X and iOS operating systems
and their respective APIs, Cocoa and Cocoa Touch.

Let's take a look of the advantages of Objective-C.

First of all, code, written in Objective-C and C, can be used in Swift,
but not vice versa. For example, the existing solutions and libraries
were all created in Objective-C but are used in Swift without any issues.
You have probably seen Apple’s Top 25 Favourite iOS Apps 2015. If you dig deeper
you will be surprised to learn how many of them were built in Objective-C
despite of the company’s dedication to Swift. Still it is not the reason
to refuse using this modern programming language from Apple.
On the contrary, they encourage developers to switch to Swift.
Maybe 2016 will end up with a much more different rank.

Secondly, C++ code cannot be used in Swift. It is, however, possible to use Objective-C++.

Thirdly, Objective-C can be compiled into static libraries and dynamic frameworks,
while Swift can be compiled only into dynamic frameworks.

The syntax of Objective-C is stable while Swift syntax is still improving.
Nevertheless, Swift 3.0 promises to bear breaking changes that will not be followed
by other great ones. This means that developers will have to convert code
from Swift 2.1 to Swift 2.2 and only then to Swift 3.0. Experience shows that
sometimes converting runs unsmoothly and it is necessary to rewrite pieces of code.
Not the most convenient thing, of course, but we have to understand that Swift
is a new programming language and updates are unavoidable.

Apps written in Swift before Version 3.0 will be 10-20 Mb bigger in size than
the ones built in Objective-C. Everything depends on architecture here.
The reason is that all Swift runtime libraries have to be included in an app if it
contains at least 1 line of Swift code. The creators promise to stabilize API in
Swift 3.0 so there will be no need doing this anymore.

Swift 2.1 compiler is unstable. It crashes sometimes which never happens
in Objective-C. Swift code is compiled slower but this is because Swift compiler
knows more about variables, than Objective-C compiler. It analyzes code deeper,
in particular type inference, generics, and protocols with associated types.
This issue will be possibly and hopefully solved in Swift 2.2. By the way,
97\% of all existing crashes are fixed already.

Xcode does not support refactoring of Swift code. It should be done manually.
This feature is available only for Objective-C.

Now let us consider the Swift language.

Swift is a modern programming language that is fast, safe and interactive.
Swift is a powered programming tool that is used to develop applications for iOS,
OS X, and watchOS. Using swift one can create codes easily as it has many feature
that have fun and it is interactive, the syntax is short and expressive.
With the use of Swift one can run apps lightning fast speed as it works
side-by-side with Objective-C.

Following are the main features of Swift language.

\textbf{Error handling model.}
Swift provide an advance error handling model that is clear, expressive syntax
of exceptional handling. Developers can create their own custom error names
so that it can describe clear error cases.

\textbf{Syntax Improvements.}
Now using Swift you can create more expressive codes as it has improved
a lot across the language. The SDK has employed new Objective-C features such as
generics and nullability annotation to keep the codes clean and safe.

\textbf{Open Source.}
Later this year Swift will be termed as open source. It is a unique combination
of power, elegance and safety that has ability to move entire software
industry forward.

\textbf{Modern.}
Consider it to be the latest research of programming language, combined with
a vast experience of developing Apple platforms. Named parameters brought
forward from Objective-C are expressed in a clean syntax that makes APIs
in Swift even easier to read and maintain.

\textbf{Interactive playgrounds.}
Using playgrounds one can write swift codes with incredibly simple and fun.
Results are appeared immediately as you type the codes. The result view
can display graphics, lists of results, or graphs of a value over time.
You can open the Timeline Assistant to watch a complex view evolve and animate,
great for experimenting with new UI code, or to play an animated SpriteKit
scene as you code it.

\textbf{Designed for Safety.}
It eliminates the entire class of unsafe code. Variables are always initialized
before use, arrays and integers are checked for overflow, and memory is
managed automatically.

\textbf{Fast and powerful.}
Swift is a fast compiling and executable programming language. It use high
performance LLVM compiler, Swift code is transformed into optimized native code
that gets the most out of modern hardware.


\subsection{Databases}

\textbf{Core Data}

\textbf{Realm}

\subsection{Addition tools}



\pagebreak