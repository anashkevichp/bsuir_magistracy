\section[Основные понятия многопоточного программирования]
{ОСНОВНЫЕ ПОНЯТИЯ МНОГОПОТОЧНОГО \\ ПРОГРАММИРОВАНИЯ}

Многопоточное программирование --- тема, актуальная для любого разработчика в
контексте разработки современных приложений. Ведь нет ничего хуже приложения,
парализованного из-за блокировки главного потока, потому что, как правило,
именно в главном потоке происходит обработка действий пользователя, а также
отрисовка пользовательского интерфейса. Соответственно, при блокировке
главного потока приложение перестаёт отвечать на запросы пользователя,
существенно понижая желание пользователя повторно воспользоваться приложением.

При разработке современных приложений с графическим интерфейсом пользователя,
даже если разработчик напрямую не создаёт потоки в приложении, как правило,
оно всё равно будет многопоточным, потому что системные библиотеки используют
дополнительные потоки для выполнения запланированных операций вне главного потока.
Аналогичная ситуация обстоит с операционной системой \textit{iOS}, устанавливаемой
на смартфоны компании \textit{Apple iPhone} и планшеты \textit{iPad} того же производителя.
Любое приложение в рамках этой операционной системы запускается с пятью потоками
по умолчанию: с одним главным потоком и четырьмя дополнительными потоками
параллельного выполнения (\textit{concurrent}) с различным приоритетом, позволяя
разработчику использовать один из этих потоков для выполнения асинхронных
операций.

\subsection{Основные понятия}

Рассмотрим понятийный аппарат многопоточного программирования.
Имеет смысл начать с рассмотрения потока и процесса.

\textbf{Процесс} --- экземпляр программы во время выполнения, независимый объект,
которому выделены системные ресурсы (например, процессорное время и память).
Каждый процесс выполняется в отдельном адресном пространстве: один процесс
не может получить доступ к переменным и структурам данных другого.
Если процесс хочет получить доступ к <<чужим>> ресурсам,
необходимо использовать межпроцессное взаимодействие.

Каждый процесс (приложение) состоит из одного или нескольких потоков, каждый из
которых представляет собой отдельный путь выполнения с помощью кода приложения.
Приложения могут порождать дополнительные потоки, каждый из которых выполняет
код конкретной функции. Когда приложение порождает новый поток, этот поток
становится независимым органом внутри пространства процесса приложения.
Каждый поток имеет собственный стек исполнения и может общаться
с другими потоками, выполнять операции ввода/вывода.
Поскольку поток и процесс находятся в одном пространстве, все потоки
в одном приложении используют одно и то же виртуальное пространство памяти
и имеют одни и те же права доступа, как и сам процесс. Таким образом, можно
определить понятие <<поток>>.

\textbf{Поток} --- определенный способ выполнения процесса или его части,
находящийся в едином пространстве конкретного процесса (приложения).

\textbf{Очередь} (в многопоточном программировании) --- упорядоченная группа
операций на выполнение.

\textbf{Задача} (операция) --- абстрактная концепция некоторого количества
выполняемой работы.

\textbf{Синхронное выполнение операций} предполагает блокировку текущего потока
до тех пор, пока операция, вызвавшая эту блокировку, не будет завершена.
Таким образом, операция, которая стоит на очереди выполнения следующей за синхронно
выполняющейся в текущий момент операцией, начнет своё выполнение только после
завершения текущей операции.

\textbf{Асинхронное выполнение операций} не блокирует текущий поток. Таким образом,
следующие операции могут быть завершены до окончания выполнения текущей операции.


\subsection{Проблемы многопоточного программирования и способы \\ их решения}

Одной из главных проблем многопоточного программирования является
необходимость обеспечения безопасного доступа к ресурсам из разных потоков.
Для решения дааной проблемы существуют различные механизмы синхронизации.

Например, при использовании ресурса, доступного из различных потоков,
имеет смысл заблокировать доступ к нему из других потоков для того на время
чтения или записи из текущего потока, чтобы исключить возможность повреждения или
некорректного чтения ресурса, что, в свою очередь, может повлечь за собой
падение приложения.

\textbf{Критическая секция} --- участок исполняемого кода программы, в котором
производится доступ к общему ресурсу (данным или устройству), который
не должен быть одновременно использован более чем одним потоком исполнения.
Объектом критической секции может владеть только один поток,
что и обеспечивает механизм синхронизации.

\textbf{Семафор} --- объект, ограничивающий количество потоков, которые могут
войти в заданный участок кода. Семафоры используются для синхронизации
и защиты передачи данных через разделяемую память, а также для синхронизации
работы процессов и потоков.

\textbf{Мьютекс} отличается от семафора тем, что только владеющий им поток
может его освободить, то есть перевести в отмеченное состояние.
Мьютекс --- это один из вариантов семафорных механизмов для организации
взаимного исключения. Основное назначение --- организация взаимного исключения
для потоков из одного и того же или из разных процессов.
Целью мьютексов является защита данных от повреждения в результате асинхронных
изменений, однако могут порождаться другие проблемы, напрмиер, взаимная блокировка.

\textbf{Взаимная блокировка} (англ. deadlock) --- ситуация в многозадачной среде,
при которой несколько процессов находятся в состоянии бесконечного ожидания ресурсов,
занятых этими же процессами.

