\section[Стандарт реализации потоков выполнения POSIX]{%
  СТАНДАРТ РЕАЛИЗАЦИИ ПОТОКОВ \\
  ВЫПОЛНЕНИЯ POSIX}

В прошлом программные интерфейсы управления потоками,
предоставляемые различными операционными системами, существенно различались.
Этот факт значительно усложнял написание кроссплатформенных программ.
В 1995 году был разработан и опубликован стандарт
программного интерфейса потоков IEEE POSIX 1003.c.
Он представляет собой набор связанных типов, функций и констант
языка C, позволяющих управлять жизненным циклом потоков
и выполнять их синхронизацию.
На данный момент программные интерфейсы упрвления потоками,
предоставляемые практически всеми операционными системами
на базе UNIX, являются POSIX-совместимыми~\cite{pthreads_programming}.

В соответствии со стандартом, программный интерфейс управления потоками
POSIX (Pthreads) описан в заголовочном файле \textit{<pthreads.h>}.
Следует отметить, что данный заголовочный файл не является частью
стандартной библиотеки языка C.

\subsection{Управление жизненным циклом потока}

Рассмотрим основные функции и типы данных, определенные стандартом Pthreads
и предназначенные для управления потоками.

Функция \textit{pthread\_create()} предназначена для создания и запуска
нового потока.
Она принимает следующие аргументы:
\begin{itemize}
  \item \textit{pthread\_t* thread} --- идентификатор потока;
  \item \textit{const pthread\_attr\_t* attr} --- атрибуты потока;
  \item \textit{void *(*start\_routine)(void*)} --- функция, предназначенная для
    выполнения в новом потоке;
  \item \textit{void* arg} --- аргумент, передаваемый в \textit{start\_routine}.
\end{itemize}
Данная функция возвращает нулевое значение в случае успеха
или код ошибки в противном случае.

Функция \textit{pthread\_exit()} предназначена для завершения вызывающего потока.
Она принимает параметр \textit{void* value\_ptr}, предназначенный для передачи
возвращаемого значения в поток, ожидающий завершения.

Функция \textit{pthread\_join()} используется для ожидания вызывающим потоком
завершения работы указанного потока. Она принимает следующие параметры:
\begin{itemize}
  \item \textit{pthread\_t thread} --- идентификатор потока,
    завершение которого мы собираемся ожидать;
  \item \textit{void** value\_ptr} --- значение переменной \textit{value\_ptr},
    переданное завершившимся потоком в соответствующий вызов \textit{pthread\_exit()}.
\end{itemize}

Функции \textit{pthread\_attr\_init()} и \textit{pthread\_attr\_destroy()}
предназначены для инициализации и удаления структуры атрибутов
потока соответственно. Структура атрибутов потока используется для
задания свойств создаваемого потока с помощью функций, описанных ниже.

С помощью функций \textit{pthread\_attr\_(get/set)detachstate()} можно указать,
будет ли поток создан в состоянии \textit{joinable} или \textit{detached}.
Разница между этими двумя типами потоков заключается в том,
что вызовы функций \textit{pthread\_join()} и \textit{pthread\_detach()}
в отношении \textit{detached} потока возвращают приводят к ошибке.

Функция \textit{pthread\_detach()} используется для перевода указанного \textit{joinable}
потока в состояние \textit{detached}.

Функции \textit{pthread\_attr\_*stack*()} предназначены для управления
параметрами стека запускаемого потока. Дело в том, что значения параметров стека
не являются стандартизованными, а поэтому могут различаться на различных ОС.
Каждая из этих функций принимает на вход структуру атрибутов потока и связанный параметр:
\begin{itemize}
  \item \textit{pthread\_attr\_(get/set)stacksize()} --- получение/установка размера стека
    создаваемого потока;
  \item \textit{pthread\_attr\_(get/set)stackaddr()} --- получение/установка стартового адреса
    стека создаваемого потока.
\end{itemize}

Поскольку все \textit{pthread\_attr*()}-функции осуществляют доступ к структуре
атрибутов потока, их вызов должен осуществляться перед созданием потока.

Функция \textit{pthread\_self()} позволяет вызывающему потоку получить свой идентификатор,
a \textit{pthread\_equal()} позволяет сравнить пару идетификаторов потока.
Функция \textit{pthread\_once()} принимает пару аргументов:
\begin{itemize}
  \item структуру синхронизации \textit{once\_control};
  \item функцию \textit{init\_routine}, подлежащую запуску в отдельном потоке.
\end{itemize}

Она устроена таким образом, что её многократные вызовы с одинаковым набором аргументов
приводят к тому, что её функция-аргумент вызывается в отдельном потоке лишь один (первый) раз.
На рисунке~\ref{lst:pthreads_basic} представлен простейший пример управления POSIX-потоком.
\lstinputlisting[
    basicstyle=\scriptsize\ttfamily,
    caption=Пример управления потоком POSIX,
    language={C},
    label=lst:pthreads_basic,
]{lst/pthreads_basic.c}

Здесь главный поток, выполняющий функцию \textit{main()},
создаёт, запускает \textit{NUM\_THREADS} потоков, выполняющих функцию \textit{task()},
и ожидает их завершения. Функция \textit{task()} выполняет циклический вывод
аргумента типа \textit{char} на консоль. На рисунке~\ref{lst:pthreads_basic_output}
приведен участок вывода данной программы.
\lstinputlisting[
    basicstyle=\scriptsize\ttfamily,
    caption=Пример несинхронизированного вывода,
    label=lst:pthreads_basic_output,
]{lst/pthreads_basic_output.lst}

Нетрудно заметить, что результат вывода представляет собой случайную
последовательность символов \textit{'a'} и \textit{'b'}.
Это происходит вследствие того, что ОС выполняет переключение между
выполняемыми потоками в случайные моменты времени, вызывая тем самым прерывание
последовательности одинаковых символов, печатаемых данным потоком.
Для того, чтобы вывод символов осуществлялся в определенном неслучайном порядке,
необходимо выполнять синхронизацию потоков, рассматриваемую в следующем подразделе.

\subsection{Примитивы синхронизации}

На практике часто возникает необходимость синхронизации работы набора потоков.

% Здесь в функции \textit{main()} выполняется создание потока,
% выполняющего функцию \textit{printHello()} с использованием
% аргумента, представляющего собой структуру \textit{thread\_data}.
% Следует отметить, что в тех случаях, когда исполнение функции
% потока заканчивается, выход из потока выполняется автоматически.



% *** Управление жизненным циклом потока
% *** Примитивы синхронизации