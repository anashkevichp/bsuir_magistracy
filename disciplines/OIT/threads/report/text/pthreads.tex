\section[Стандарт реализации потоков выполнения POSIX]{%
  СТАНДАРТ РЕАЛИЗАЦИИ ПОТОКОВ \\
  ВЫПОЛНЕНИЯ POSIX}

В прошлом программные интерфейсы управления потоками,
предоставляемые различными операционными системами, существенно различались.
Этот факт значительно усложнял написание кроссплатформенных программ.
В 1995 году был разработан и опубликован стандарт
программного интерфейса потоков IEEE POSIX 1003.c.
Он представляет собой набор связанных типов, функций и констант
языка C, позволяющих управлять жизненным циклом потоков
и выполнять их синхронизацию.
На данный момент программные интерфейсы упрвления потоками,
предоставляемые практически всеми операционными системами
на базе UNIX, являются POSIX-совместимыми~\cite{pthreads_programming}.

В соответствии со стандартом, программный интерфейс управления потоками
POSIX (Pthreads) описан в заголовочном файле \texttt{<pthreads.h>}.
Следует отметить, что данный заголовочный файл не является частью
стандартной библиотеки языка C.

\subsection{Управление жизненным циклом потока}

Рассмотрим основные функции и типы данных, определенные стандартом Pthreads
и предназначенные для управления потоками.

Функция \texttt{pthread\_create()} предназначена для создания и запуска
нового потока.
Она принимает следующие аргументы:
\begin{itemize}
  \item \texttt{pthread\_t* thread} --- идентификатор потока;
  \item \texttt{const pthread\_attr\_t* attr} --- атрибуты потока;
  \item \texttt{void *(*start\_routine)(void*)} --- функция, предназначенная для
    выполнения в новом потоке;
  \item \texttt{void* arg} --- аргумент, передаваемый в \texttt{start\_routine}.
\end{itemize}
Данная функция возвращает нулевое значение в случае успеха
или код ошибки в противном случае.

Функция \texttt{pthread\_exit()} предназначена для завершения вызывающего потока.
Он имеет параметр \texttt{void* value\_ptr}, предназначенный для передачи
возвращаемого значения в поток, ожидающий завершения.

Функции \texttt{pthread\_attr\_init} и \texttt{pthread\_attr\_destroy}
предназначены для инициализации и удаления структуры атрибутов
потока соответственно.
На рисунке~\ref{lst:pthreads_args} представлен простейший пример создания потока.
\lstinputlisting[
    caption=Создание потока POSIX,
    language={C},
    label=lst:pthreads_args,
]{lst/pthreads_args.lst}

Здесь в функции \texttt{main()} выполняется создание потока,
выполняющего функцию \texttt{printHello()} с использованием
аргумента, представляющего собой структуру \texttt{thread\_data}.
Следует отметить, что в тех случаях, когда исполнение функции
потока заканчивается, выход из потока выполняется автоматически.



% *** Управление жизненным циклом потока
% *** Примитивы синхронизации