\section[Семафоры POSIX]{СЕМАФОРЫ POSIX}

Наряду с мьютексами и условными переменными,
существует еще один широко известный примитив синхронизации,
называемый семафором.
Он был предложен голландским ученым Э. Дейкстрой в 1962 году.
Семафор предназначен для ограничения числа потоков,
которые могут одновременно исполнять некоторый защищенный участок кода.

Над семафором можно производить три операции:
\begin{itemize}
  \item инициализация семафора, в ходе которой задается начальное значение счетчика;
  \item захват семафора, в ходе которого если счетчик имел положительное
    значение, то оно уменьшается на единицу, иначе поток, осуществляющий захват, блокируется;
  \item освобождение семафора, в ходе которого значение семафора увеличивается на единицу.
\end{itemize}

Нетрудно заметить, что мьютекс представляет собой двоичный семафор.
В некоторых реализациях семафоров используется очередь --- потоки,
ожидающие освобождения семафора, будут проходить через семафор
именно в том порядке, в котором они вызывали пытались его захватить~\cite{wiki_semaphore}.

Семафоры не являются частью стандарта Pthreads, а описаны в другом стандарте
POSIX.1b, Real-time extensions (IEEE Std 1003.1b-1993).
Тем не менее, их можно использовать совместно с POSIX threads.
Программный интерфейс работы с семафорами описан в заголовочном файле \textit{<semaphore.h>}.

Семафор POSIX имеет тип \textit{sem\_t}. Для работы с ним предусмотрены следующие функции:
\begin{itemize}
\item \textit{sem\_init} --- инициализация семафора;
\item \textit{sem\_wait} --- захват семафора;
\item \textit{sem\_post} --- освободение семафора;
\item \textit{sem\_getvalue} --- получение текущего значения счетчика семафора;
\item \textit{sem\_destroy} --- уничтожение семафора;
\end{itemize}

На рисунке~\ref{lst:semaphore_basic} приведена модификация исходной версии программы,
использующая семафор для синхронизации многопоточного вывода в консоль.
\pagebreak

\lstinputlisting[
    caption=Использование семафора POSIX,
    language={C},
    label=lst:semaphore_basic,
]{lst/semaphore_basic.c}

На рисунке~\ref{lst:semaphore_basic_output} приведен пример
вывода модифицированной версии программы.

\lstinputlisting[
    caption=Пример работы программы~\ref{lst:semaphore_basic},
    language={C},
    label=lst:semaphore_basic_output,
]{lst/semaphore_basic_output.lst}

Можно утверждать, что вывод программы~\ref{lst:semaphore_basic},
использующей семафор, эквивалентен выводу программы~\ref{lst:pthreads_mutex},
использующей мьютекс, поскольку и в том и другом случае вывод потоков является
синхронизированным, но его порядок, вообще говоря, не определен.