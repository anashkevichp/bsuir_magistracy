\section*{ВВЕДЕНИЕ}
\addcontentsline{toc}{section}{Введение}

В данной работе рассматривается одна из основных задач теории
дифференциальных уровнений -- задача Коши. Приведена формулировка задачи,
а также краткая классификация методов её решения.

Решение задачи Коши является крайне актуальным вопросом, потому что
многие она достаточно часто возникает при анализе процессов,
определяемых дифференциальным законом эволюции и начальным состоянием
(математическим выражением которых и являются уравнение и начальное
условие), например расмотрение термодинамических законов, вычисление
размера вклада в банке после определенного количества лет, и~т.~д.

Более подробно рассмотрен
метод Хемминга решения задачи Коши, приведены его основные формулы, а также
преимущества и недостатки по сравнению с другими миетодами.

\pagebreak
