\section[Метода Хемминга решения задачи Коши]
{МЕТОД ХЕММИНГА РЕШЕНИЯ ЗАДАЧИ КОШИ}

\subsection{Описание метода}

Метод Хэмминга (Хемминга, англ. Hamming's m.) решения задачи Коши является
\textit{многошаговым}, работающий по типу \textit{<<предиктор-корректор>>}.
В данном методе используются следующие формулы.

Формула прогноза:
$$
    y^{0}_{i+1} = y_{i-3} + \dfrac{4}{3} h (2 y'_i - y'_{i-1} + 2 y'_{i-2}).
$$

Формула уточнения прогноза:
$$
    \overline{y}^{0}_{i+1} = y^{0}_{i+1} + \dfrac{112}{121} (y_i - y^{(0)}_i), \hspace{1cm} [\overline{y}^{0}_{i+1}]' = f(x_{i+1}, \overline{y}^{0}_{i+1}).
$$

Формула коррекции:
$$
    y_{i+1} = \dfrac{1}{8} (9 y_i - y_{i-2} + 3h (f_{i+1} + 2f_i - f_{i-1})).
$$

Метод Хемминга является устойчивым, обладает четвертым порядком точности.


\subsection{Преимущества и недостатки метода}

Особенностью метода Хемминга, как и для других многошаговых методов,
можно назвать необходимость расчёта нескольких начальных значений каким-либо
одношаговым методом.

Недостатком метода можно назвать возможность попадания на продолжительные
колебания около истинного решение: такое может произойти, если шаг $h$ выбран
слишком большим.

К преимуществам метода Хемминга можно отнести то, что он позволяет
оценивать погрешности, вносимые на стадиях прогноза и коррекции и устранять их.
Благодаря простоте и устойчивости этот метод является одним из наиболее
распространенных методов прогноза и коррекции.

\pagebreak